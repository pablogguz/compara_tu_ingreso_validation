\documentclass[letterpaper,11pt,leqno]{article}
\usepackage{paper}
\usepackage{natbib}
\bibliographystyle{bibliography}

\usepackage{xcolor}
\usepackage{hyperref}
\usepackage{threeparttable}
\usepackage{natbib}
\usepackage{listings}
\usepackage{rotating}
\usepackage{float}
\usepackage{pdflscape}
\usepackage{makecell}
\usepackage{booktabs}

\definecolor{customblue}{rgb}{0,0.337,0.702}

\hypersetup{
    colorlinks   = true,
    citecolor    = customblue,
    linkcolor    = customblue,
    urlcolor     = customblue
}

\def\dotfill#1{\cleaders\hbox to #1{.}\hfill}
\makeatletter
\def\myrulefill{\leavevmode\leaders\hrule height .7ex width 1ex depth -0.6ex\hfill\kern\z@}
\makeatother

\begin{document}

\title{\Large \textsc{A simple and adaptable method to estimate income distributions in Spain}}

\author{\href{https://pablogguz.github.io/}{\large{Pablo García-Guzmán}} \\ \normalsize{EBRD}}

\date{\normalsize \today \\ \textcolor{red}{Preliminary draft -- do not cite or circulate}}

\maketitle
\begin{abstract}
\onehalfspacing
This methodology note develops a transparent and simple framework to estimate income distributions at granular geographical levels in Spain. Specifically, I leverage administrative tax records and demographic data at the census tract-level -- small areas encompassing 1,000 to 2,500 residents -- to reconstruct local income distributions assuming log-normality. National and regional income distributions are subsequently derived as population-weighted mixtures of these tract-level distributions. This approach leverages well-documented regularities in how incomes are distributed within neighborhoods, and the resulting estimates closely match observed income percentiles in validation tests. I implement this framework in \href{https://comparatuingreso.es/}{comparatuingreso.es}, a publicly available web platform that enables Spanish households to calculate their relative position within the income distribution. While the focus is on Spain, this approach can be readily adapted to other countries with comparable administrative data available.
\end{abstract}

\thispagestyle{empty}

\newpage

\setcounter{page}{1}

\section{Introduction}

The rest of this note is organized as follows. Section 2 describes the data sources, their coverage, and scope. Section 3 outlines the methodology, including the modeling of local income distributions and their aggregation. Section 4 presents validation results, and Section 5 discusses potential extensions and applications of the framework to other contexts.

\section{Data}


The data used in this project is sourced from the Spanish Statistical Office (INE) Household Income Distribution Atlas (\textit{Atlas de Distribución de Renta de los Hogares}, ADRH). This dataset combines administrative tax data with population statistics to provide detailed information about the income distribution and related socioeconomic indicators at granular geographic levels in Spain.

Spain is administratively divided into several territorial levels that structure its governance and statistical reporting. At the highest level are the \textit{Comunidades Autónomas} (Autonomous Communities), which are regions with significant legislative and executive powers granted by the Spanish Constitution. There are 17 Autonomous Communities and two Autonomous Cities (Ceuta and Melilla). These regions are further subdivided into \textit{provincias} (provinces), the primary intermediate level of governance. Each province consists of \textit{municipios} (municipalities), the fundamental local administrative units, which vary widely in size and population.

Below the municipal level, finer geographic divisions provide additional granularity. Larger municipalities are typically divided into \textit{distritos} (districts), which correspond to distinct urban or rural areas within the municipality. The smallest unit of analysis is the \textit{sección censal} (census tract), encompassing areas with populations ranging from 1,000 to 2,500 residents. Census tracts provide the finest level of geographic detail available in the ADRH.

Income data in the ADRH is derived from tax declarations submitted to the Spanish tax authorities, including the \textit{Agencia Estatal de Administración Tributaria} (AEAT) and the Foral tax authorities.\footnote{The latter operate in regions with special fiscal regimes, such as the Basque Country and Navarre.} These tax records provide detailed information on income from various sources, including wages, pensions, unemployment benefits, and other forms of revenue subject to the \text{Impuesto sobre la Renta de las Personas Físicas} (IRPF), Spain's personal income tax system. The dataset excludes non-resident income and focuses exclusively on individuals considered fiscal residents within the territory.

Demographic data in the ADRH is constructed from the \textit{Fichero Precensal de Población} (FPC), a comprehensive register derived from the municipal register (\textit{Padrón}) and other administrative sources. The FPC forms the basis of Spain’s population census and ensures alignment between demographic and income data. The temporal reference for income data corresponds to the calendar year, while demographic data is anchored to the population as registered on January 1 of the following year. Individuals residing in collective establishments, such as nursing homes, hospitals, or military barracks, are excluded from the dataset.

\paragraph{Summary statistics} The sample consists of 36,982 census tracts covering all of Spain in 2022. The average tract has 1,303 residents, though size varies considerably (SD = 663). Average net income per equivalent adult is €20,798, about 48\% higher than per capita income (€14,030). Within-tract inequality, measured by the Gini coefficient, averages 29 points across tracts. Missing rates are below 2\% for most variables, and rise slightly above 5\% for equivalized income and the Gini coefficient. Demographically, the average tract has a dependency ratio of 0.63, a mean age of 45.4 years, and 30.5\% single-person households.

\begin{table}[H]
\centering
\resizebox{0.8\textwidth}{!}{% 
\begin{threeparttable}[H]
\onehalfspacing
\centering
\captionsetup{justification=centering} 
\caption{Summary statistics}

\begin{tabular}{@{}lccccc@{}}
\toprule
  & (1) & (2) & (3) &  (4) &  (5) \\
 & Min & Max & Mean & SD & \% missing\\
\midrule
\qquad \textit{Income distribution} \\
Net income per capita &  4,996.00 & 34,765.00 & 14,030.16 &  4,193.34 &      1.64\\
Net income per equivalent adult &  8,092.00 & 55,987.00 & 20,797.51 &  6,628.08 &      5.21\\
Gini &     20.30 &     44.20 &     29.02 &      3.84 &      5.21\\
\qquad \textit{Demographics} \\
Population &      3.00 & 12,144.00 &  1,303.10 &    663.26 &      1.45\\
Dependency ratio &      0.09 &      4.00 &      0.63 &      0.17 &      1.45\\
Mean age &     27.40 &     74.70 &     45.43 &      5.52 &      1.45\\
Single-person households (\%) &      0.00 &    100.00 &     30.50 &      9.56 &      1.45\\
\midrule
No. of tracts &          36,983 \\
\midrule
\end{tabular}
\begin{tablenotes}[para,flushleft]
\footnotesize 
\textbf{Notes}: all statistics are unweighted and calculated at the census-tract level. Spanish Statistical Office (INE) Household Income Distribution Atlas and author's calculations.
\end{tablenotes}
\end{threeparttable}
}
\end{table}

To improve coverage, I conduct two simple adjustments. First, I impute net income per equivalent adult within missing tracts using net income per capita and the average provincial ratio between these two measures, calculated using population weights from non-missing observations. This adjustment reduces missing rates in equivalised income per capita from 5.20\% to 1.63\%. Second, I impute missing Gini coefficients (5.20\% of tracts) using a machine learning approach. More details can be found in Section \ref{sec:method}.

\section{Methodology}


\subsection{Background}

The study of income distributions has been a cornerstone of economic research. A key empirical regularity in this literature is that income distributions tend to approximate a log-normal distribution \citep{aitchison1957lognormal}. This pattern has been documented across different contexts and time periods, leading to its establishment as a stylized fact in economic analysis \citep{lopez2006normal}.

The theoretical foundations for log-normality rest on multiple reinforcing mechanisms. First, individual earnings are determined by the multiplicative interaction of factors like human capital, experience, local labor market conditions, and individual productivity \citep{neal2000theories}. Second, Gibrat's law of proportionate effect demonstrates how such multiplicative processes naturally generate log-normal distributions \citep{gibrat1931inegalites}. This theoretical prediction is particularly robust within demographically and geographically homogeneous groups \citep{battistin2009consumption}.
The spatial dimension adds another layer to this analysis. \cite{sampson2012great} demonstrates that residential location influences economic outcomes through multiple channels, including peer effects, social networks, and institutional quality. 

\cite{durlauf2004neighborhood} formalizes this insight, developing a theoretical framework explaining how local social interactions generate correlation in economic outcomes within neighborhoods. These spatial effects are further reinforced through local labor market networks \citep{topa2015neighborhood}, creating a mechanism where geographic proximity translates into similar economic outcomes. When populations share similar characteristics and face common local conditions, their incomes tend to follow more precisely a log-normal distribution than in more heterogeneous populations \citep{mitzenmacher2004brief}.

\paragraph{Mixtures} These insights about local homogeneity and overall heterogeneity suggest the utility of mixture distributions. \cite{parker2012measuring} demonstrates that mixtures of log-normal distributions can accurately capture both within-group and between-group inequality, providing support for approaches that combine local distributional estimates. This is particularly relevant when working with administrative data that provides detailed geographic information but may lack individual-level granularity. The mixture approach allows us to leverage the stronger log-normality at local levels while accommodating the greater complexity of aggregate distributions.

\paragraph{Limitations} While the log-normal distribution provides a good approximation, several studies have documented systematic deviations, particularly in the tails. \cite{clementi2016some} find that while the body of the income distribution is well-approximated by a log-normal, the upper tail often follows a Pareto distribution. This "double distribution" pattern suggests the need for caution when using log-normal approximations, particularly when analyzing high-income populations.

\subsection{Empirical approach}

I estimate the national income distribution by exploiting the granular structure of census tract data and the theoretical properties of income distributions. My approach leverages two key insights: (1) income distributions within small geographic units tend to follow log-normal distributions more closely than in larger areas, and (2) the overall distribution can be approximated by a mixture of these local distributions.

Consider a census tract $j$ with an observed mean income $\mu_j$ and Gini coefficient $G_j$. Under the log-normality assumption, if income $X$ follows a log-normal distribution with parameters $(\nu_j, \sigma_j^2)$, then:

\begin{equation}
\ln(X) \sim N(\nu_j, \sigma_j^2)
\end{equation}

The relationship between these parameters and the observed statistics is thus given by:

\begin{equation}
\mu_j = \exp\left(\nu_j + \frac{\sigma_j^2}{2}\right)
\end{equation}

\begin{equation}
G_j = 2\Phi\left(\frac{\sigma_j}{\sqrt{2}}\right) - 1
\end{equation}

where $\Phi(\cdot)$ is the standard normal cumulative distribution function. From the observed Gini coefficient, I can recover $\sigma_j$:

\begin{equation}
\sigma_j = \sqrt{2}\Phi^{-1}\left(\frac{G_j + 1}{2}\right)
\end{equation}

Given $\sigma_j$ and $\mu_j$, the distribution for census tract $j$ is fully identified, , as $\nu_j$ can be solved analytically. I then estimate the national distribution as a population-weighted mixture of these local log-normal distributions. For a given income level $x$, the density is given by:

\begin{equation}
f(x) = \sum_{j=1}^{J} w_j f_j(x|\nu_j,\sigma_j^2)
\end{equation}

where $w_j$ is tract $j$'s population share and $f_j(\cdot|\nu_j,\sigma_j^2)$ is the log-normal density function with parameters $(\nu_j,\sigma_j^2)$.

To calculate percentiles of the national distribution, I solve:

\begin{equation}
p = F(q_p) = \sum_{j=1}^{J} w_j \Phi\left(\frac{\ln(q_p) - \nu_j}{\sigma_j}\right)
\end{equation}

where $q_p$ is the $p^{th}$ percentile and $F(\cdot)$ is the cumulative distribution function of the mixture. I use numerical root-finding methods to find the value of $q_p$ that satisfies $F(q_p) - p = 0$ within a bounded interval.

This approach has several advantages. First, it respects the theoretical properties of income distributions at local levels where populations are more homogeneous. Second, it preserves tract-level moments while allowing for heterogeneity across tracts. Third, it provides a flexible framework for calculating any distributional statistic of interest.

\paragraph{Missing outcomes} For census tracts with missing Gini coefficients (approximately 5.2\% of the sample), I estimate them using machine learning methods. Specifically, I train an XGBoost model using the following demographic predictors: dependency ratio (ratio of population under 18 and over 65 to working-age population), mean age, percentage of single-person households, mean logged equivalised income, and province fixeed effects. Model performance is validated using a 5-fold cross-validation procedure. The XGBoost model is trained using default hyperparameters without grid search or additional tuning. Based on the cross-validation results, the XGBoost model demonstrates superior predictive performance compared to a baseline OLS model.

\paragraph{Validation} To validate the log-normality assumption at the tract level, I compare observed distributional statistics with their theoretical counterparts under log-normality. For each tract $j$, I first recover the parameters of the theoretical log-normal distribution $(\nu_j, \sigma_j)$ using the observed mean income and Gini coefficient. Then, I calculate two key predicted statistics: the P80/P20 ratio and the median income. Under log-normality, these are given by:

\begin{equation}
\text{P80/P20}_j = \exp(\sigma_j(\Phi^{-1}(0.8) - \Phi^{-1}(0.2)))
\end{equation}

\begin{equation}
\text{P50}_j = \exp(\nu_j)
\end{equation}

To assess the fit, I estimate binned regressions of the form:

\begin{equation}
y_j^{\text{predicted}} = \alpha + \beta y_j^{\text{observed}} + \epsilon_j
\end{equation}



\section{Results}



%Bibliography
\newpage
\singlespacing
\bibliographystyle{authordate3}
\bibliography{bib.bib}

\end{document}

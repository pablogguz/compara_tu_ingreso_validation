\documentclass[letterpaper,11pt,leqno]{article}
\usepackage{paper}
\usepackage{natbib}
\bibliographystyle{bibliography}

\usepackage{xcolor}
\usepackage{hyperref}
\usepackage{threeparttable}
\usepackage{natbib}
\usepackage{listings}
\usepackage{rotating}
\usepackage{float}
\usepackage{pdflscape}
\usepackage{makecell}
\usepackage{booktabs}

\definecolor{customblue}{rgb}{0,0.337,0.702}

\hypersetup{
    colorlinks   = true,
    citecolor    = customblue,
    linkcolor    = customblue,
    urlcolor     = customblue
}

\def\dotfill#1{\cleaders\hbox to #1{.}\hfill}
\makeatletter
\def\myrulefill{\leavevmode\leaders\hrule height .7ex width 1ex depth -0.6ex\hfill\kern\z@}
\makeatother

\begin{document}

\title{\Large \textsc{A simple and adaptable method to estimate income distributions in Spain}}

\author{\href{https://pablogguz.github.io/}{\large{Pablo García-Guzmán}} \\ \normalsize{EBRD}}

\date{\normalsize \today \\ \textcolor{red}{Preliminary draft -- do not cite or circulate}}

\maketitle
\begin{abstract}
\onehalfspacing
This methodology note develops a transparent and simple framework to estimate income distributions at granular geographical levels in Spain. Specifically, I leverage administrative tax records and demographic data at the census tract-level -- small areas encompassing 1,000 to 2,500 residents -- to reconstruct local income distributions assuming log-normality. National and regional income distributions are subsequently derived as population-weighted mixtures of these tract-level distributions. This approach leverages well-documented regularities in how incomes are distributed within neighborhoods, and the resulting estimates closely match observed income percentiles in validation tests. I implement this framework in \href{https://comparatuingreso.es/}{comparatuingreso.es}, a publicly available web platform that enables Spanish households to calculate their relative position within the income distribution. While the focus is on Spain, this approach can be readily adapted to other countries with comparable administrative data available.
\end{abstract}

\thispagestyle{empty}

\newpage

\setcounter{page}{1}

\section{Introduction}

Understanding the distribution of income across society is crucial for economic analysis and policy design, yet its measurement remains challenging. In this paper, I propose a novel method to estimate the national income distribution by exploiting granular geographic data. The approach combines tract-level income statistics with the theoretical properties of income distributions to reconstruct the entire density. Using detailed administrative data from Spain covering 36,982 census tracts, I show that local income distributions are well-approximated by log-normal distributions, with predicted statistics matching observed values within 5\% on average. This finding allows me to estimate the full national distribution as a population-weighted mixture of tract-specific log-normals. The method provides a flexible framework for analyzing income distributions when individual data is unavailable but detailed geographic information exists.

The rest of this note is organized as follows. Section \ref{sec:data} describes the data sources, their coverage, and scope. Section \ref{sec:method} outlines the methodology, including the modeling of local income distributions, their aggregation and the calculation of the relevant validation metrics. Section \ref{sec:results} presents validation results, and Section 5 discusses potential extensions and applications of the framework to other contexts.

\section{Data}\label{sec:data}


The data used in this project is sourced from the Spanish Statistical Office (INE) Household Income Distribution Atlas (\textit{Atlas de Distribución de Renta de los Hogares}, ADRH). This dataset combines administrative tax data with population statistics to provide detailed information about the income distribution and related socioeconomic indicators at granular geographic levels in Spain.

Spain is administratively divided into several territorial levels that structure its governance and statistical reporting. At the highest level are the \textit{Comunidades Autónomas} (Autonomous Communities), which are regions with significant legislative and executive powers granted by the Spanish Constitution. There are 17 Autonomous Communities and two Autonomous Cities (Ceuta and Melilla). These regions are further subdivided into \textit{provincias} (provinces), the primary intermediate level of governance. Each province consists of \textit{municipios} (municipalities), the fundamental local administrative units, which vary widely in size and population.

Below the municipal level, finer geographic divisions provide additional granularity. Larger municipalities are typically divided into \textit{distritos} (districts), which correspond to distinct urban or rural areas within the municipality. The smallest unit of analysis is the \textit{sección censal} (census tract), encompassing areas with populations ranging from 1,000 to 2,500 residents. Census tracts provide the finest level of geographic detail available in the ADRH.

Income data in the ADRH is derived from tax declarations submitted to the Spanish tax authorities, including the \textit{Agencia Estatal de Administración Tributaria} (AEAT) and the Foral tax authorities.\footnote{The latter operate in regions with special fiscal regimes, such as the Basque Country and Navarre.} These tax records provide detailed information on income from various sources, including wages, pensions, unemployment benefits, and other forms of revenue subject to the \text{Impuesto sobre la Renta de las Personas Físicas} (IRPF), Spain's personal income tax system. The dataset excludes non-resident income and focuses exclusively on individuals considered fiscal residents within the territory.

Demographic data in the ADRH is constructed from the \textit{Fichero Precensal de Población} (FPC), a comprehensive register derived from the municipal register (\textit{Padrón}) and other administrative sources. The FPC forms the basis of Spain’s population census and ensures alignment between demographic and income data. The temporal reference for income data corresponds to the calendar year, while demographic data is anchored to the population as registered on January 1 of the following year. Individuals residing in collective establishments, such as nursing homes, hospitals, or military barracks, are excluded from the dataset.

\paragraph{Summary statisticts}.


\begin{table}[H]
\centering
\resizebox{0.8\textwidth}{!}{% 
\begin{threeparttable}[H]
\onehalfspacing
\centering
\captionsetup{justification=centering} 
\caption{Covid impact}

\begin{tabular}{@{}lccccc@{}}
\toprule
  & (1) & (2) & (3) &  (4) &  (5) \\
 & Min & Max & Mean & SD & \% missing\\
\midrule
\qquad \textit{Income distribution} \\
Net income per capita &  4,996.00 & 34,765.00 & 14,030.16 &  4,193.34 &      1.63\\
Net income per equivalent adult &  8,092.00 & 55,987.00 & 20,797.51 &  6,628.08 &      5.20\\
Gini &     20.30 &     44.20 &     29.02 &      3.84 &      5.20\\
\qquad \textit{Demographics} \\
Population &      3.00 & 12,144.00 &  1,303.10 &    663.26 &      1.45\\
Dependency ratio &      0.09 &      4.00 &      0.63 &      0.17 &      1.45\\
Mean age &     27.40 &     74.70 &     45.43 &      5.52 &      1.45\\
Single-person households (%) &      0.00 &    100.00 &     30.50 &      9.56 &      1.45\\
\midrule
No. of tracts &          36,982 \\
\midrule
\end{tabular}
\begin{tablenotes}[para,flushleft]
\footnotesize 
\textbf{Source}: Spanish Statistical Office (INE) Household Income Distribution Atlas and author's calculations.
\end{tablenotes}
\end{threeparttable}
}
\end{table}

\section{Methodology}\label{sec:method}


\subsection{Background}

The study of income distributions has been a cornerstone of economic research. A long-standing empirical regularity in the literature is that income distributions tend to approximate log-normality \citep{aitchison1957lognormal}. This stylized fact is supported by theoretical and empirical foundations. At its core, log-normality arises from the multiplicative interaction of economic factors—such as human capital, local labor market conditions, and productivity—which determine individual incomes \citep{neal2000theories}. Gibrat's Law of Proportionate Effect \citep{gibrat1931inegalites} formalizes a theoretical foundation for log-normality through multiplicative random growth, showing that when the growth rate of a variable is independent of its initial size and the logarithm of the growth rate is independent and identically distributed over time with finite variance, the resulting distribution tends to converge to log-normality. Empirically, log-normal densities have been shown to approximate very well per capita income distributions in large cross-country panels \citep{lopez2006normal}.

More concretely, the multiplicative processes through which log-normality arises operate most strongly within demographically homogeneous groups where individuals are subject to similar economic shocks \citep{battistin2009consumption}. The literature of neighborhood effects offers a complementary explanation for why such homogeneity in income-generating processes exists at local levels. Specifically, neighborhood effects reinforce homogeneity within local populations through contextual influences (e.g., average neighborhood income or quality of local services) and endogenous spillovers, which emerge from behavioral interactions within the neighborhood, such as peer influences or social norms \citep{manski1993identification, durlauf1996neighborhoods}. These mechanisms create reinforcing feedback loops that generate correlated income trajectories within neighborhoods -- for instance, through shared information about job opportunities, similar human capital accumulation patterns, or common responses to local economic shocks.  The resulting interaction structures generate strong within-neighborhood homogeneity while maintaining between-neighborhood heterogeneity \citep{durlauf2004neighborhood}. This pattern provides a theoretical rationale  for estimating aggregate income distributions as mixtures of local-level distributions.

%Relatedly, a more recent literature on place effects has documented that the environment where children grow up has substantial causal impacts on their economic outcomes later in life \cite{chynplace}. Using experimental variation from the Moving to Opportunity program, \cite{chetty2016effects} show that childhood exposure to better neighborhoods significantly increases college attendance and earnings in adulthood.

The log-normal distribution, while providing a good approximation for much of the income distribution, exhibits systematic deviations in the tails. Since Pareto's \citeyearpar{pareto1896cours} original work, research has shown that top incomes follow a power law rather than a log-normal decay \citep{gabaix2016power}. Similarly, studies using detailed administrative tax data have documented that the upper tail of income distributions across countries and time periods are better described by a Pareto distribution \citep{atkinson2011top}. This departure from log-normality at high incomes is important for accurately measuring top income inequality, but modeling the precise behavior of the upper tail is beyond the scope of this note.

\subsection{Empirical approach}

I estimate the national income distribution by exploiting the granular structure of census tract data and the theoretical properties of income distributions. My approach leverages two key insights: (1) income distributions within small geographic units tend to follow log-normal distributions more closely than in larger areas, and (2) the overall distribution can be approximated by a mixture of these local distributions.

Consider a census tract $j$ with an observed mean income $\mu_j$ and Gini coefficient $G_j$. Under the log-normality assumption, if income $Y$ follows a log-normal distribution with parameters $(\nu_j, \sigma_j^2)$, then:

\begin{equation}
\ln(Y) \sim N(\nu_j, \sigma_j^2)
\end{equation}

The relationship between these parameters and the observed statistics is thus given by:

\begin{equation}
\mu_j = \exp\left(\nu_j + \frac{\sigma_j^2}{2}\right)
\end{equation}

\begin{equation}
G_j = 2\Phi\left(\frac{\sigma_j}{\sqrt{2}}\right) - 1
\end{equation}

where $\Phi(\cdot)$ is the standard normal cumulative distribution function. From the observed Gini coefficient, I can recover $\sigma_j$:

\begin{equation}
\sigma_j = \sqrt{2}\Phi^{-1}\left(\frac{G_j + 1}{2}\right)
\end{equation}

Given $\sigma_j$ and $\mu_j$, the distribution for census tract $j$ is fully identified, , as $\nu_j$ can be solved analytically. I then estimate the national distribution as a population-weighted mixture of these local log-normal distributions. For a given income level $y$, the density is given by:

\begin{equation}
f(y) = \sum_{j=1}^{J} w_j f_j(x|\nu_j,\sigma_j^2)
\end{equation}

where $w_j$ is tract $j$'s population share and $f_j(\cdot|\nu_j,\sigma_j^2)$ is the log-normal density function with parameters $(\nu_j,\sigma_j^2)$.

To calculate percentiles of the national distribution, I solve:

\begin{equation}
p = F(q_p) = \sum_{j=1}^{J} w_j \Phi\left(\frac{\ln(q_p) - \nu_j}{\sigma_j}\right)
\end{equation}

where $q_p$ is the $p^{th}$ percentile and $F(\cdot)$ is the cumulative distribution function of the mixture. I use numerical root-finding methods to find the value of $q_p$ that satisfies $F(q_p) - p = 0$ within a bounded interval.

This approach has several advantages. First, it respects the theoretical properties of income distributions at local levels where populations are relatively homogeneous. Second, it preserves tract-level moments while allowing for heterogeneity across tracts. Third, it provides a flexible framework that can be adapted and extended to calculate any distributional statistic of interest.

\paragraph{Missing outcomes} For census tracts with missing Gini coefficients (approximately 5.2\% of the sample), I estimate them using machine learning methods. Specifically, I train an XGBoost model using the following demographic predictors: dependency ratio (ratio of population under 18 and over 65 to working-age population), mean age, percentage of single-person households, mean logged equivalised income, and province fixed effects. Model performance is validated using a 5-fold cross-validation procedure. The XGBoost model is trained using default hyperparameters without grid search or additional tuning. Results are shown in Table \ref{perf}. Based on the cross-validation results, the XGBoost model demonstrates superior predictive performance compared to a baseline OLS model, as illustrated by a 11\% lower RMSE.

\begin{table}[H]
\centering
\resizebox{0.7\textwidth}{!}{% 
\begin{threeparttable}[H]
\onehalfspacing
\centering
\captionsetup{justification=centering} 
\caption{Performance comparison between OLS and XGBoost models}\label{perf}
\begin{tabular}{lccc}
\hline
Model    & RMSE & MAE & MAPE (\%) \\
\hline
OLS                & 3.15          & 2.41         & 8.32              \\
XGBoost (CV)       & 2.81          & 2.17         & 7.49              \\
\hline
Relative improvement over OLS (\%) & 10.8\% & 10.0\% & 10.0\% \\
\hline
\end{tabular}
\begin{tablenotes}[para,flushleft]
\scriptsize 
\textbf{Notes}: This table compares the performance of two predictive models—OLS and XGBoost—on predicting missing Gini coefficients using demographic predictors. RMSE represents the root mean square error, MAE is the mean absolute error, and MAPE is the mean absolute percentage error. Relative improvements in each metric are calculated as the percentage reduction of the error metric when using XGBoost compared to OLS. Results are based on a 5-fold cross-validation procedure for the XGBoost model.
\end{tablenotes}
\end{threeparttable}
}
\end{table}


\paragraph{Validation} To validate the log-normality assumption at the tract level, I compare observed distributional statistics with their theoretical counterparts under log-normality. For each tract $j$, I first recover the parameters of the theoretical log-normal distribution $(\nu_j, \sigma_j)$ using the observed mean income and Gini coefficient. Then, I calculate two key predicted statistics: the P80/P20 ratio and the median income. Under log-normality, these are given by:

\begin{equation}
\text{P80/P20}_j = \exp(\sigma_j(\Phi^{-1}(0.8) - \Phi^{-1}(0.2)))
\end{equation}

\begin{equation}
\text{P50}_j = \exp(\nu_j)
\end{equation}

To assess the fit, I estimate OLS regressions of the form:

\begin{equation}
\hat{y}_j = \alpha + \beta y_j + \epsilon_j
\end{equation}

where $\hat{y}_j$ is the predicted value under log-normality and $y_j$ is the observed value.

\section{Results}



%Bibliography
\newpage
\singlespacing
\bibliographystyle{authordate3}
\bibliography{bib.bib}

\end{document}

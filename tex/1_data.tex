
The data used in this project is sourced from the Spanish Statistical Office (INE) Household Income Distribution Atlas (\textit{Atlas de Distribución de Renta de los Hogares}, ADRH). This dataset combines administrative tax data with population statistics to provide detailed information about the income distribution and related socioeconomic indicators at granular geographic levels in Spain.

Spain is administratively divided into several territorial levels that structure its governance and statistical reporting. At the highest level are the \textit{Comunidades Autónomas} (Autonomous Communities), which are regions with significant legislative and executive powers granted by the Spanish Constitution. There are 17 Autonomous Communities and two Autonomous Cities (Ceuta and Melilla). These regions are further subdivided into \textit{provincias} (provinces), the primary intermediate level of governance. Each province consists of \textit{municipios} (municipalities), the fundamental local administrative units, which vary widely in size and population.

Below the municipal level, finer geographic divisions provide additional granularity. Larger municipalities are typically divided into \textit{distritos} (districts), which correspond to distinct urban or rural areas within the municipality. The smallest unit of analysis is the \textit{sección censal} (census tract), encompassing areas with populations ranging from 1,000 to 2,500 residents. Census tracts provide the finest level of geographic detail available in the ADRH.

Income data in the ADRH is derived from tax declarations submitted to the Spanish tax authorities, including the \textit{Agencia Estatal de Administración Tributaria} (AEAT) and the Foral tax authorities.\footnote{The latter operate in regions with special fiscal regimes, such as the Basque Country and Navarre.} These tax records provide detailed information on income from various sources, including wages, pensions, unemployment benefits, and other forms of revenue subject to the \text{Impuesto sobre la Renta de las Personas Físicas} (IRPF), Spain's personal income tax system. The dataset excludes non-resident income and focuses exclusively on individuals considered fiscal residents within the territory.

Demographic data in the ADRH is constructed from the \textit{Fichero Precensal de Población} (FPC), a comprehensive register derived from the municipal register (\textit{Padrón}) and other administrative sources. The FPC forms the basis of Spain’s population census and ensures alignment between demographic and income data. The temporal reference for income data corresponds to the calendar year, while demographic data is anchored to the population as registered on January 1 of the following year. Individuals residing in collective establishments, such as nursing homes, hospitals, or military barracks, are excluded from the dataset.
